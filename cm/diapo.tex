\documentclass{beamer}
  \usepackage[utf8]{inputenc}
  \usetheme{Warsaw}
  \graphicspath{images/}
  \usepackage{amsmath} 
  \usepackage{amssymb}
  \title{udev \& sysfs : fonctionnement et administration}
  \author{UFR des Sciences Versailles - M2 SeCReTS}
  \institute{CAUMES Clément \& DEBROUASSE Kevin \& \\ HEQUET Jonathan \& Mehdi MTALSI-MERIMI}
  \date{Lundi 25 Novembre 2019}
  \begin{document}

  \begin{frame}
  \titlepage
  \end{frame}

\section{Définitions}

\subsection{Notions de noyau}

 \begin{frame}
	\begin{block}{Kernel} 
	Le kernel d'un système d'exploitation est le logiciel permettant : 
	\begin{itemize}
		\setbeamertemplate{itemize item}[circle]
		\item la gestion des systèmes de fichiers
		\item la communication entre les logiciels et les périphériques connectés à l'appareil
		\item la gestion et la communication entre les processus
	\end{itemize}
	\end{block}

	\begin{block}{Pilote} 
		Un pilote est un programme qui permet au système d'exploitation d'interagir avec un périphérique.
	\end{block}


\end{frame}

\subsection{/dev}
 \begin{frame}
	\begin{block}{Périphérique} 
		Tout élément dans un OS Linux est représenté par un fichier. De cette manière, les périphériques sont stockés sous forme de fichiers dans le dossier /dev.
		Ces fichiers vont donc permettre l’accès aux périphériques et sont appelés device nodes. 
	\end{block}
    \begin{block}{Device Node}
    	Un device node est un point d’entrée vers le noyau caractérisé par un type (bloc ou char) et deux nombres: le major et le minor. Ce triplé définit de façon unique quel périphérique matériel est accédé via ce fichier. Le majeur permet au noyau de savoir quel driver doit gérer le périphérique et le mineur permet au driver de savoir quel périphérique parmi ceux qu’il gère est utilisé. 
    \end{block}
	\begin{exampleblock}{ls -l /dev} 
		 Permet de voir les différents node devices de la machine.
	\end{exampleblock}
\end{frame}

\section{Historique avant udev}

\subsection{makedev}
\begin{frame}
\begin{block}{Qu'est ce que makedev?} 
	Dans les premières versions de Linux, les numéros de major et minor sont codés dans le noyau. Pour pouvoir modifier ces numéros, il fallait modifier le script MAKEDEV, ce qui était très contraignant en cas d'erreur. De plus, l'espace de valeurs possibles pour les numéros major et minor était trop petit. 
	Il fallait donc trouver une autre solution pour les versions suivantes de Linux.
\end{block}
\end{frame}

\subsection{devfs}
\begin{frame}
\begin{block}{Qu'est ce que devfs?} 
	Devfs est un système de fichiers contenant les device nodes et dont les noeuds sont créés par les pilotes des périphériques lors de leur détection. Cependant, il y avait toujours certaines limites (espace de numéros trop petit notamment).
\end{block}
\end{frame}

\section{Définitions de udev et sysfs}

\subsection{udev}

\begin{frame}
\begin{block}{Définition} 
	udev est un gestionnaire de périphériques du dossier /dev. Il va créer des nœuds dynamiquement pour les périphériques connectés au système. 
	Udev détecte lorsqu'un nouveau périphérique est connecté. 
\end{block}
\begin{exampleblock}{udevadm info -a -p} 
	
	\tiny{\$  udevadm monitor -k -p
		
		monitor will print the received events for:
		KERNEL - the kernel uevent
		
		KERNEL[8414.073713] add      /devices/pci0000:00/0000:00:0b.0/usb1/1-1 (usb)\newline
		ACTION=add\newline
		DEVPATH=/devices/pci0000:00/0000:00:0b.0/usb1/1-1\newline
		SUBSYSTEM=usb\newline
		DEVNAME=/dev/bus/usb/001/027\newline
		DEVTYPE=usb\_device\newline
		PRODUCT=1e3d/2093/100\newline
		TYPE=0/0/0\newline
		BUSNUM=001\newline
		DEVNUM=027\newline
		SEQNUM=2455\newline
		MAJOR=189\newline
		MINOR=26\newline}
\end{exampleblock}
 
\end{frame}

\begin{frame}
\begin{exampleblock}{Interprétation des propriétés importantes} 
Les propriétés dépendent du type exact de périphérique mais certaines propriétés sont toujours présentes:
\begin{itemize}
	\setbeamertemplate{itemize item}[circle]
	\item ACTION : Le type d’événement à traiter
	\item MAJOR, MINOR : Les numéro majeur et mineur du périphérique concerné. 
    \item SEQNUM : un numéro croissant pour ordonner les événements, 
    \item SUBSYSTEM : Le sous-système noyau ayant causé l’événement, 
	\item DEVPATH : Le fichier dans /sys correspondant au périphérique. 
\end{itemize}
\end{exampleblock}
\end{frame}


\subsection{sysfs}

\begin{frame}
\begin{block}{Fonctions} 
	\begin{itemize}
		\setbeamertemplate{itemize item}[circle]
		\item Sysfs est un système de fichier virtuel, c'est à dire couche d'abstraction au dessus du système de fichier physique
		\item Il introduit en même temps que udev avec le noyau 2.6 de linux 
		\item Sysfs exporte depuis l'espace noyau vers l'espace utilisateur les informations 
		sur les périphériques du système. Ainsi, il va créer un dossier associé au système de fichiers contenant une 
		suite de fichiers représentant les attributs du périphérique en question. 
	\end{itemize}
\end{block}

\begin{exampleblock}{ls /sys/block/sdb} 
	Permet de visualiser les fichiers représentant les attributs du premier périphérique USB connecté.
\end{exampleblock}
\end{frame}

\section{Ecriture de règles}

\section{Administration}

\subsection{Log définition}

\begin{frame}
\begin{block}{Définition} 
	Un log (ou logging) est un fichier permettant de stocker un historique des événements sur une machine. C'est donc un journal de bord qui est utilisé dans l'administration système pour garder une trace de ce qui s'est passé (pas forcément des incidents).
\end{block}


\begin{alertblock}{Informations utiles pour les logs}
	\begin{itemize}
		\setbeamertemplate{itemize item}[circle]
		\item Date et heure de l'action
		\item Identification de l'action 
		\item Auteur de l'action (dans l'idéal)
		\item Identification de l'outil permettant d'effectuer l'action
	\end{itemize}
\end{alertblock}
\end{frame}

\subsection{udev et les logs}
\begin{frame}
\begin{block}{Utilité de udev et de sysfs pour l'administration} 
	udev permet de lancer des scripts lors de la connexion et déconnexion de périphériques sur la machine. \newline On peut donc écrire dans un fichier (de log) afin d'identifier l'action (connexion ou déconnexion) et le périphérique. 
\end{block}


\begin{exampleblock}{Extrait du résultat généré par une règle udev pour l'administration}
	\tiny{
		Fri Nov 15 11:23:02 CET 2019 - CAUMES Generic\_Flash\_Disk\_CCCB1104231104350952973414-0:0 sdb1 : connexion \newline
		Fri Nov 15 11:23:04 CET 2019 - HEQUET Verbatim\_STORE\_N\_GO\_070126D061196C46-0:0 sdc1 : connexion \newline
		Fri Nov 15 11:23:09 CET 2019 - CAUMES Generic\_Flash\_Disk\_CCCB1104231104350952973414-0:0 sdb1 : deconnexion \newline
		Fri Nov 15 11:23:11 CET 2019 - HEQUET Verbatim\_STORE\_N\_GO\_070126D061196C46-0:0 sdc1 : deconnexion \newline
	}
\end{exampleblock}
\end{frame}

\section{References}

 \begin{frame}
\begin{block}{Références} 
	\begin{itemize}
		\setbeamertemplate{itemize item}[circle]
		\item \footnotesize \url{doc.ubuntu-fr.org/udev} \normalsize
		\item \footnotesize \url{linuxconfig.org/tutorial-on-how-to-write-basic-udev-rules-in-linux} \normalsize
		\item \footnotesize \url{www.linuxembedded.fr/2015/05/une-introduction-a-udev/} \normalsize
	\end{itemize}
\end{block}
\end{frame}
      
\end{document}
