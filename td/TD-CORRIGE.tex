\documentclass[11pt]{article}
\usepackage[utf8]{inputenc}
\usepackage[T1]{fontenc}
\usepackage[french]{babel}
\usepackage[top=1.8cm, bottom=1.8cm, left=1.8cm, right=1.8cm]{geometry}
\usepackage[linktocpage,colorlinks=false]{hyperref}
\usepackage{graphicx}
\usepackage{epsfig}
\usepackage{amssymb}
\usepackage{amsmath}
\usepackage{array}
\usepackage{subfig}
\usepackage{multicol}
\usepackage{caption}
\usepackage{algorithm}
\usepackage{color}
\usepackage{algorithmic}
\usepackage{listings}
\usepackage{url}
\usepackage{fullpage}
\usepackage{color}
\usepackage[table]{xcolor}
\usepackage{listings}

\definecolor{darkWhite}{rgb}{0.94,0.94,0.94}

\lstset{
	aboveskip=3mm,
	belowskip=-2mm,
	backgroundcolor=\color{darkWhite},
	basicstyle=\footnotesize,
	breakatwhitespace=false,
	breaklines=true,
	captionpos=b,
	commentstyle=\color{gray},
	deletekeywords={...},
	escapeinside={\%*}{*)},
	extendedchars=true,
	framexleftmargin=16pt,
	framextopmargin=3pt,
	framexbottommargin=6pt,
	frame=tb,
	keepspaces=true,
	keywordstyle=\color{blue},
	language=C,
	literate=
	{²}{{\textsuperscript{2}}}1
	{⁴}{{\textsuperscript{4}}}1
	{⁶}{{\textsuperscript{6}}}1
	{⁸}{{\textsuperscript{8}}}1
	{€}{{\euro{}}}1
	{é}{{\'e}}1
	{è}{{\`{e}}}1
	{ê}{{\^{e}}}1
	{ë}{{\¨{e}}}1
	{É}{{\'{E}}}1
	{Ê}{{\^{E}}}1
	{û}{{\^{u}}}1
	{ù}{{\`{u}}}1
	{â}{{\^{a}}}1
	{à}{{\`{a}}}1
	{á}{{\'{a}}}1
	{ã}{{\~{a}}}1
	{Á}{{\'{A}}}1
	{Â}{{\^{A}}}1
	{Ã}{{\~{A}}}1
	{ç}{{\c{c}}}1
	{Ç}{{\c{C}}}1
	{õ}{{\~{o}}}1
	{ó}{{\'{o}}}1
	{ô}{{\^{o}}}1
	{Õ}{{\~{O}}}1
	{Ó}{{\'{O}}}1
	{Ô}{{\^{O}}}1
	{î}{{\^{i}}}1
	{Î}{{\^{I}}}1
	{í}{{\'{i}}}1
	{Í}{{\~{Í}}}1,
	morekeywords={*,...},
	numbers=left,
	numbersep=10pt,
	numberstyle=\tiny\color{black},
	rulecolor=\color{black},
	showspaces=false,
	showstringspaces=false,
	showtabs=false,
	stepnumber=1,
	stringstyle=\color{black},
	tabsize=4,
	title=\lstname,
}



\hypersetup{
    colorlinks=true,
    breaklinks=true,
    urlcolor=red,
}
\parskip=5pt

\title{\huge{\textbf TD \\UDEV et SYSFS : Ecriture de règles et administration}}
\date{Lundi 25 novembre 2019}

\begin{document}

\maketitle

\subsection*{Exercice 1}

\begin{enumerate}
	\item Effectuer la commande suivante "udevadm monitor -k -p". Connecter ensuite un périphérique USB. Déconnecter enfin ce périphérique. Que remarquez-vous? 
	
	\lstinputlisting[caption={udevadm monitor -k -p}, language=SH]{./reponses/q1a} \bigskip 
	
	La commande permet de détecter la connexion ou la déconnexion de n'importe quel périphérique. C'est bien udev qui gère les périphériques.\bigskip
	
	\item Reconnecter la clé USB et exécuter la commande "udevadm info -a -p /sys/block/sdb". Qu'en déduisez-vous par rapport à la commande de la question précédente? 
	
	\lstinputlisting[caption={udevadm info -a -p /sys/block/sdb}, language=SH]{./reponses/q1b} \bigskip
	
	Quand on veut les informations d'un périphérique déjà connecté, on utilise cette commande. Elle permet de visualiser certains attributs. 
	Ces attributs sont des attributs générés par sysfs. Sysfs est un système de fichiers virtuel qui va récupérer les attributs de chaque périphérique et créer leurs 
	attributs correspondants. Dans les questions suivantes, on pourra visualiser où ces attributs sont créés.
	
	\item Faire la commande "df -h" et retrouver avec le système de fichiers associé à la clé USB. 
	Parcourir le dossier /sys/block/sd[a-z] en fonction de ce que vous avez trouvé à la question précédente. Que remarquez-vous?
	
	\lstinputlisting[caption={df -h}, language=SH]{./reponses/q1c1} \bigskip
	
	\lstinputlisting[caption={sys/block/sdb}, language=SH]{./reponses/q1c2} \bigskip
	
	On peut donc voir ici que sysfs exporte depuis l'espace noyau vers l'espace utilisateur les informations 
	sur les périphériques du système. Ainsi, il va créer un dossier associé au système de fichiers contenant une 
	suite de fichiers représentant les attributs du périphérique en question. Ainsi, c'est udev qui va 
	interpréter les fichiers générés par sysfs pour donner ces attributs à l'utilisateur. Cela permet donc 
	de créer des règles qui vont s'appliquer en fonction des attributs des périphériques. \newpage
	
	
	\item Effectuer la commande "sudo /sbin/blkid -o udev -p /dev/sdb1".
	
	\lstinputlisting[caption={/sbin/blkid}, language=SH]{./reponses/q1d} \bigskip
	
	Cette commande est intéressante car elle permet de récupérer le nom de la clé notamment. 
	
\end{enumerate}

\subsection*{Exercice 2}

Il serait intéressant d'avoir un fichier de log (pour toujours avoir une trace utile pour l'administration) 
des différentes connexions et déconnexions de clés USB, ainsi que le montage/démontage de nouvelles partitions
sur le disque dur. 

\begin{enumerate}
	\item Créer une nouvelle règle dans laquelle on déclare de nouvelles variables d'envrionnement représentant le 
	chemin du fichier de log ainsi que celui du script qui écrira dans le fichier de log. 
	\item Créer une ligne qui va détecter l'ajout d'une nouvelle partition (sda[0-9]) ou d'une connexion de clés USB (sd[b-z][0-9])
	et qui va lancer le script d'écriture de log avec des paramètres en ligne de commande (nom du périphérique, numéro de série et chemin du fichier de log).
	\item Créer une ligne qui va détecter la suppression d'une partition ou la déconnexion d'une clé USB (de la même manière que la question précédente)
	
	\lstinputlisting[caption={11-log.rules}, language=SH]{./reponses/q2a} \bigskip
	
	\item Créer le script d'écriture de log avec toutes ces informations sans oublier la date et l'heure de l'action.
	
	\lstinputlisting[caption={create\_log.sh}, language=SH]{./reponses/q2b} \bigskip
	
	\item Etablir une règle udev qui réalise les questions antérieures sans passer par un script (indice : utiliser "echo").
	
	\lstinputlisting[caption={11-log-v2.rules}, language=SH]{./reponses/q2c} \bigskip

\end{enumerate}

\subsection*{Exercice 3}

Sur certaines versions de linux, lorsque l'on connecte un périphérique USB, un point de montage 
est automatiquement créé dans /home/usr/Desktop/. Faites de même avec une règle udev. 
(Indice : utiliser /usr/bin/systemd-mount au lieu de /usr/bin/mount : \url{https://wiki.archlinux.org/index.php/Udev})

\lstinputlisting[caption={12-automount.rules}, language=SH]{./reponses/q3} \bigskip



\end{document}