\documentclass[11pt]{article}
\usepackage[utf8]{inputenc}
\usepackage[T1]{fontenc}
\usepackage[french]{babel}
\usepackage[top=1.8cm, bottom=1.8cm, left=1.8cm, right=1.8cm]{geometry}
\usepackage[linktocpage,colorlinks=false]{hyperref}
\usepackage{graphicx}
\usepackage{epsfig}
\usepackage{amssymb}
\usepackage{amsmath}
\usepackage{array}
\usepackage{subfig}
\usepackage{multicol}
\usepackage{caption}
\usepackage{algorithm}
\usepackage{color}
\usepackage{algorithmic}
\usepackage{listings}
\usepackage{url}
\usepackage{fullpage}
\usepackage{color}
\usepackage[table]{xcolor}
\usepackage{listings}

\definecolor{darkWhite}{rgb}{0.94,0.94,0.94}

\lstset{
	aboveskip=3mm,
	belowskip=-2mm,
	backgroundcolor=\color{darkWhite},
	basicstyle=\footnotesize,
	breakatwhitespace=false,
	breaklines=true,
	captionpos=b,
	commentstyle=\color{gray},
	deletekeywords={...},
	escapeinside={\%*}{*)},
	extendedchars=true,
	framexleftmargin=16pt,
	framextopmargin=3pt,
	framexbottommargin=6pt,
	frame=tb,
	keepspaces=true,
	keywordstyle=\color{blue},
	language=C,
	literate=
	{²}{{\textsuperscript{2}}}1
	{⁴}{{\textsuperscript{4}}}1
	{⁶}{{\textsuperscript{6}}}1
	{⁸}{{\textsuperscript{8}}}1
	{€}{{\euro{}}}1
	{é}{{\'e}}1
	{è}{{\`{e}}}1
	{ê}{{\^{e}}}1
	{ë}{{\¨{e}}}1
	{É}{{\'{E}}}1
	{Ê}{{\^{E}}}1
	{û}{{\^{u}}}1
	{ù}{{\`{u}}}1
	{â}{{\^{a}}}1
	{à}{{\`{a}}}1
	{á}{{\'{a}}}1
	{ã}{{\~{a}}}1
	{Á}{{\'{A}}}1
	{Â}{{\^{A}}}1
	{Ã}{{\~{A}}}1
	{ç}{{\c{c}}}1
	{Ç}{{\c{C}}}1
	{õ}{{\~{o}}}1
	{ó}{{\'{o}}}1
	{ô}{{\^{o}}}1
	{Õ}{{\~{O}}}1
	{Ó}{{\'{O}}}1
	{Ô}{{\^{O}}}1
	{î}{{\^{i}}}1
	{Î}{{\^{I}}}1
	{í}{{\'{i}}}1
	{Í}{{\~{Í}}}1,
	morekeywords={*,...},
	numbers=left,
	numbersep=10pt,
	numberstyle=\tiny\color{black},
	rulecolor=\color{black},
	showspaces=false,
	showstringspaces=false,
	showtabs=false,
	stepnumber=1,
	stringstyle=\color{black},
	tabsize=4,
	title=\lstname,
}



\hypersetup{
    colorlinks=true,
    breaklinks=true,
    urlcolor=red,
}
\parskip=5pt

\title{\huge{\textbf {TD} \\UDEV et SYSFS : Ecriture de règles et administration}}
\date{Lundi 25 novembre 2019}
\author{CAUMES Clément - DEBROUASSE Kevin - HEQUET Jonathan - MERIMI MTALSI Mehdi}

\begin{document}

\maketitle

\subsection*{Exercice 1}

\begin{enumerate}
	\item Effectuer la commande suivante "udevadm monitor -k -p". Connecter ensuite un périphérique USB. Déconnecter enfin ce périphérique. Que remarquez-vous? 
	
	\item Reconnecter la clé USB et exécuter la commande "udevadm info -a -p /sys/block/sdb". Qu'en déduisez-vous par rapport à la commande de la question précédente? 
	
	\item Faire la commande "df -h" et retrouver avec le système de fichiers associé à la clé USB. 
	Parcourir le dossier /sys/block/sd[a-z] en fonction de ce que vous avez trouvé à la question précédente. Que remarquez-vous?
	
	\item Effectuer la commande "sudo /sbin/blkid -o udev -p /dev/sdb1" (Aide pour la compréhension : \url{http://manpages.ubuntu.com/manpages/trusty/fr/man8/blkid.8.html}).
	
\end{enumerate}

\subsection*{Exercice 2}

Il serait intéressant d'avoir un fichier de log (pour toujours avoir une trace utile pour l'administration) 
des différentes connexions et déconnexions de clés USB, ainsi que le montage/démontage de nouvelles partitions
sur le disque dur. 

\begin{enumerate}
	\item Créer une nouvelle règle dans laquelle on déclare de nouvelles variables d'envrionnement représentant le 
	chemin du fichier de log ainsi que celui du script qui écrira dans le fichier de log. 
	\item Créer une ligne qui va détecter l'ajout d'une nouvelle partition (sda[0-9]) ou d'une connexion de clés USB (sd[b-z][0-9])
	et qui va lancer le script d'écriture de log avec des paramètres en ligne de commande (nom du périphérique, numéro de série et chemin du fichier de log).
	\item Créer une ligne qui va détecter la suppression d'une partition ou la déconnexion d'une clé USB (de la même manière que la question précédente)
	
	\item Créer le script d'écriture de log avec toutes ces informations sans oublier la date et l'heure de l'action.
	
	\item Etablir une règle udev qui réalise les questions antérieures sans passer par un script (indice : utiliser "echo").

\end{enumerate}

\subsection*{Exercice 3}

Sur certaines versions de linux, lorsque l'on connecte un périphérique USB, un point de montage 
est automatiquement créé dans /home/usr/Desktop/. Faites de même avec une règle udev. 
(Indice : utiliser /usr/bin/systemd-mount au lieu de /usr/bin/mount : \url{https://wiki.archlinux.org/index.php/Udev})

\subsection*{Exercice 4}

Ce qui pourrait être utile de réaliser grâce aux règles udev et les attributs de sysfs serait de faire un backup d'un dossier important de la machine lors de la connection d'une certaine clé USB (celle qui vous appartient de préférence). 

\begin{enumerate}
	\item Réaliser une règle udev permettant de faire un backup du dossier de votre choix présent sur votre machine sur n'importe quelle clé USB qui se connecte (Aide : Faire attention à la synchronisation des différentes commandes ; Utiliser la commande rsync pour faire un backup).

	\item Modifier la règle précédente afin de faire le backup uniquement sur la clé USB qui vous appartient (Indice : Utiliser le numéro de série de votre clé).		
	
\end{enumerate}

\end{document}